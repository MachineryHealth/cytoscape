\documentstyle[11pt,doublespace,epsf]{article}

\topmargin=-45pt
\oddsidemargin=0cm
\evensidemargin=0cm
\textheight=23.7cm
\textwidth=16cm


\begin{document}
\setcounter{section}{9}
\setcounter{page}{18}
\section{On Low Frequencies of AUG Trinucleotides in front of Start Codons}
In this section, we focus on the skipped AUGs that are close to start
codons.

Ribosomes ignore AUGs that should not serve as start codons by leaky 
scanning and they select appropriate AUGs as start codons. But do
ribosomes confuse to select appropriate AUG as start codon if two AUGs
are close to each other? We investigated this hypothesis from
4 approaches. That is frequencies of AUG trinucleotides around start codons,
rate of leaky scanning when two AUGs are close to each other,
consensus sequences when two AUGs are close, and relationship between
translation initiation mechanisms and evolutions.


\subsection{Frequencies of AUG trinucleotides around Start Codons}
 

Frequencies of AUG trinucleotides around start codons are calculated
to find whether the frequencies decrease around start codons.

\vspace{2ex}
\noindent
\begin{tabular}{|l|}
\hline
Hypothesis:\\
\hline
\end{tabular}

Ribosomes confuse to select appropriate AUG as 
start codon if two AUGs are close to each other. Thus, the frequencies
of AUG trinucleotides around start codons are relatively low to
prevent the wrong selection of the AUGs.

\vspace{2ex}
\noindent
\begin{tabular}{|l|}
\hline
Data analysis:\\
\hline
\end{tabular}

Computer analysis was done to analyze the
frequencies of AUG trinucleotides around start codons. Except for
bacteria, mRNA data was used. For bacteria, DNA data was used, because
mRNA data for bacteria is very few. The frequency of position \(F_{p}\) is
calculated as follows.

\begin{displaymath}
F_{p} = \frac{Number\:of\:sequences\:that\:has\:AUG\:in\:position\:p}
      {Number\:of\:sequence\:data\:that\:has\:data\:for\:position\:p}
\end{displaymath}

And then smoothing was done for every 3 positions to reduce too much 
fluctuations. Because sequences with AUG trinucleotides located
upstream of start 
codons and in the same reading frame(every 3 positions from start
codon)are excluded for the precise analyses(described in
\ref{method_3}), we multiplied the smoothed  
value by \(\frac{3}{2}\) for upstream.

If the hypothesis holds, then the frequency of AUG trinucleotide gets
lower as the position in mRNA gets closer to start codon.
If not, the frequencies of AUG trinucleotides upstream will be constant for
upstream, and the frequencies of AUG trinucleotides downstream will
also be constant.

\vspace{2ex}
\noindent
\begin{tabular}{|l|}
\hline
Results:\\
\hline
\end{tabular}

\begin{figure}
\epsfile{file=pri_atg.ps,scale=0.40}
\epsfile{file=rod_atg.ps,scale=0.40}\\
\epsfile{file=mam_atg.ps,scale=0.40}
\epsfile{file=vrt_atg.ps,scale=0.40}\\
\epsfile{file=inv_atg.ps,scale=0.40}
\epsfile{file=bct_atg.ps,scale=0.40}
\caption{Frequencies of AUG trinucleotides around start codons}
\label{atg_freq}
\end{figure}

The results(figure \ref{atg_freq}) show that there is a general
tendency to have fewer  
AUG trinucleotides in front of start codons for all taxonomical
groups.
In eucaryotes, the frequencies of AUGs just downstream of start codon is low,
comparing to the frequencies far downstream, but the frequencies are
much higher comparing to ones in front of start codons.
In other words, the frequency drastically increases at the position
 of start codon for
all eucaryotes. But in bacteria, the frequencies of AUG trinucleotides
in positions just several nucleotides downstream of start codon is still low.

\vspace{2ex}
\noindent
\begin{tabular}{|l|}
\hline
Discussion:\\
\hline
\end{tabular}

The general tendency of the low frequency near start
codon in eucaryotes is consistent
with the hypothesis. The reason that the frequency drastically
increases at the position of start codon for all eucaryotes is
presumably
because according to the scanning model in eucaryotes,
ribosome scans mRNA from 5' end to downstream, searching for AUG.
If the AUG trinucleotide is located upstream of start codon, 
leaky scanning must work to prevent the initiation from this
inappropriate AUG. But if this mechanism is not reliable, 
those AUG trinucleotides should not exist upstream of start codons.  
But once translation starts
from start codon, AUG trinucleotide downstream of start codon has no 
influence on translation initiation.
Thus, this may be the reason that the frequencies of AUG
trinucleotides upstream of start codon is low and the frequencies
 of AUG trinucleotides downstream of start codon
is relatively high comparing to ones in front of start codons.

In bacteria, however, it is thought that ribosome binds directly to 
the translation initiation site instead of scanning mRNA from upstream
to downstream. Thus, we suggest that in bacteria, 
ribosome is likely to mistake nearby downstream AUGs for start codon
as well as nearby upstream AUGs.


\subsection{Rate of Leaky Scanning and Distance between 2 AUGs}


We have investigated the relationship between distances of two AUGs
and the rate of leaky scanning.

\vspace{2ex}
\noindent
\begin{tabular}{|l|}
\hline
Hypothesis:\\
\hline
\end{tabular}

When two AUGs are close to each other, ribosome confuse to select
one of AUGs as start codon. This phenomenon has influence on leaky
scanning. 
Thus rate of leaky scanning changes when two AUGs are close to each other.

\vspace{2ex}
\noindent
\begin{tabular}{|l|}
\hline
Data analysis:\\
\hline
\end{tabular}

The rate of leaky scanning when two AUGs are separated by \(n\) bases
(\(RLK_{n}\)) is calculated as follows.

\begin{small}
\begin{displaymath}
RLK_{n} = \frac{Number\:of\:leaky\:scanning\:observed\:when\:two\:AUGs\:
are\:separated\:by\:n\:bases}{Number\:of
\:cases\:that\:there\:is\:AUG\:trinucleotide\:n\:bases\:
upstream/downstream\:of\:start\:codon}
\end{displaymath}
\end{small}

\vspace{2ex}
\noindent
\begin{tabular}{|l|}
\hline
Results:\\
\hline
\end{tabular}

\begin{figure}
\begin{center}
\epsfile{file=leak_bun.ps,scale=0.40}
\end{center}
\caption{Relationship between leaky scanning and distance between two AUGs}
\label{leak_bun}
\end{figure}

The rate of leaky scanning fluctuates(figure \ref{leak_bun}), but this holds
not only when 
two AUGs are close to each other, but also when two AUGs are separated
far from each other.

While the first AUG is selected as a
starting AUG for the most of the time, we found that the second AUG is
frequently selected by leaky scanning if the two AUG's are
separated by \(3n+2\) bases where \(n\) is an integer.  

\vspace{2ex}
\noindent
\begin{tabular}{|l|}
\hline
Discussion:\\
\hline
\end{tabular}

The distance of two AUGs does have influence on rate of leaky
scanning, but we cannot observe any remarkable features when two AUGs
are close to each other. Thus, the hypothesis is not confirmed.

The periodic peak of the rate is presumably due to the following
hypothesis(figure \ref{3n2}).
Suppose that AUGs are separated by \(3n+2\) bases and translation
starts from first AUGs. The nucleotide pattern ``ua'' and ``ug''
just before second AUG trinucleotide and ``a'' just after the second 
AUG will make stop codons in frame, disturbing the farther translation
downstream. Thus, translation initiation from first AUG when second
AUG is separated by \(3n+2\) bases restricts these nucleotide patterns
around second AUGs. This restriction may make the rate of leaky scanning 
higher.

\begin{figure}
\begin{picture}(400,120)
\put(0,73){\line(1,0){50}}
\put(50,70){AUG}
\put(73,73){\line(1,0){200}}
\put(273,70){\underline{taA}UG}
\put(305,73){\line(1,0){50}}


\put(0,43){\line(1,0){50}}
\put(50,40){AUG}
\put(73,43){\line(1,0){200}}
\put(273,40){\underline{tgA}UG}
\put(305,43){\line(1,0){50}}



\put(0,13){\line(1,0){50}}
\put(50,10){AUG}
\put(73,13){\line(1,0){210}}
\put(282,10){A\underline{UGa}}
\put(311,13){\line(1,0){44}}

\put(150,80){$3n+2$ bases}
\put(240,81){\shortstack[c]{\small Stop codons in\\ the reading frames}}
\end{picture}

\caption{When two AUGs are separated by \(3n+2\) base pairs, stop
codons are likely to be in the same reading frame.}
\label{3n2}
\end{figure}








\subsection{Tendencies When Two AUGs are Close to Each Other}

The frequencies of AUG trinucleotides around start codons
are low from our results. Thus, existing AUGs around start codons 
may have some special features to prevent or promote leaky scanning in 
order to initiate translations from the appropriate AUG. So the
following hypothesis was made.

\vspace{2ex}
\noindent
\begin{tabular}{|l|}
\hline
Hypothesis:\\
\hline
\end{tabular}

 In case where two AUGs are close to each other, strong
consensus sequence exists to prevent or to promote leaky scanning.

\vspace{2ex}
\noindent
\begin{tabular}{|l|}
\hline
Data analysis:\\
\hline
\end{tabular}

Calculation of the entropy of positions which is 3 bases upstream from 
AUG trinucleotides were done for the cases where
two AUGs are separated by specific number of bases.
Note that position -3 is thought to be the important position for the
translation initiation.
This calculation was done for the following 4 kinds of positions.\\ 
\\

\noindent
{\large Normal Scanning}\\
\begin{picture}(400,50)
\put(0,23){\line(1,0){50}}
\put(50,20){AUG}
\put(73,23){\line(1,0){60}}
\put(132,20){AUG}
\put(155,23){\line(1,0){244}}

\put(50,33){\vector(1,0){300}}
\put(50,38){Translation}

\put(40,10){\vector(0,1){8}}
\put(122,10){\vector(0,1){8}}

\put(20,2){Position A}
\put(102,2){Position B}

\end{picture}\\ \\

\noindent
{\large Leaky Scanning}\\
\begin{picture}(400,50)
\put(0,23){\line(1,0){50}}
\put(50,20){AUG}
\put(73,23){\line(1,0){60}}
\put(132,20){AUG}
\put(155,23){\line(1,0){244}}

\put(132,33){\vector(1,0){217}}
\put(132,38){Translation}

\put(40,10){\vector(0,1){8}}
\put(122,10){\vector(0,1){8}}

\put(20,2){Position C}
\put(102,2){Position D}

\end{picture}\\ \\


\begin{enumerate}
\item Position A: Position  -3 in normal scanning.
\item Position B: Position which is 3 bases upstream from AUG trinucleotide
located downstream of start codon.
\item Position C: Position which is 3 bases upstream from AUG
trinucleotide
located upstream of start codon.
\item Position D: Position -3 in leaky scanning.
\end{enumerate}

Note that if strong consensus is observed in position A or D, that may be
the strong signal to initiate translation from that position.
If strong consensus is observed in postion B or C, that may be the
strong signal to prevent initiation from that position.

Because data for
invertebrate are less, we analyzed only vertebrates.

\vspace{2ex}
\noindent
\begin{tabular}{|l|}
\hline
Results:\\
\hline
\end{tabular}

\begin{figure}
\begin{center}
\epsfile{file=m3entabcd.ps,scale=0.40}
\end{center} 
\caption{Entropies of the positions that are 3 bases upstream from AUG 
trinucleotides}
\label{m3ent}
\begin{picture}(400,0)
\put(150,53){\tiny 2}
\end{picture}
\end{figure}

Figure \ref{m3ent} shows entropy of 4 kinds of positions in vertebrates.
The entropy of position -3 in general is 1.47 and
the entropy of the position which is 3 bases upstream
from the skipped AUGs in general is 2.0(not normalized according to GC
content).  
As there were not much data, entropy value tend to be low. But
entropies of position B and position C tend to be high comparing with
position A and position D.
And from this figure, we can observe that when two AUGs are separated
 by two bases, entropy of position -3 is very low(See position A in
figure \ref{m3ent}).

\begin{table}
\begin{center}
\epsfile{file=atg_2_atg.ps,scale=0.70}
\end{center}
\caption{Nucleotide distributions around two AUGs that are exactly
2 base pair apart.Normal scanning (above) and leaky scanning (below).}
\label{atg_2_atg}
\end{table}

Table \ref{atg_2_atg} shows nucleotide distribution around two AUG's that are
two bases apart in vertebrates.  
The first table is for the cases where the
first AUG is selected, and the second table is
for the cases where the second AUG is selected, skipping the first.
In case where starting AUG is followed by second AUG with 2 bases in between, 
ACC[AUG]G is observed as 
a very "strong" consensus sequence around the starting AUG. Furthermore,
there are no thymine or cytosine in position -3. If the tendency of this
position is same as that of the vertebrate mRNAs in general, 
the possibility that this
occurs is less than 1\%. 


\vspace{2ex}
\noindent
\begin{tabular}{|l|}
\hline
Discussion:\\
\hline
\end{tabular}

As the entropies of position B and C is generally high, we suggest
that there is no strong consensus sequence at these positions to
prevent translation from non-starting AUG even when two AUGs are close 
to each other. But according to entropy of position A and position D,
there is consensus sequence at these positions to promote translation
from start codons. And we suggest that position -3 is important to
avoid leaky scanning, especially when two AUGs are separated by two bases.










\subsection{Relationship between initiation mechanism and evolution}

In this subsection, investigation about whether there is any
relationship between evolution and translation initiation mechanism
will be conducted.

\vspace{2ex}
\noindent
\begin{tabular}{|l|}
\hline
Hypothesis:\\
\hline
\end{tabular}

There is a relationship between the evolution and distance between
start codon and upstream AUG trinucleotide.

\vspace{2ex}
\noindent
\begin{tabular}{|l|}
\hline
Data analysis:\\
\hline
\end{tabular}

We have calculated the average distances between a start codon and the nearest
non-starting ATG trinucleotide to the upstream and to the
downstream. This time, we have used DNA data rather than mRNA data in
GenBank because a large amount of data are required for precise
analyses and there are more sequences as DNA data than as mRNA data.
For the sake of comparison, distances between  start codons and other
trinucleotides consisting of 'A','T','G' are measured.

\vspace{2ex}
\noindent
\begin{tabular}{|l|}
\hline
Results:\\
\hline
\end{tabular}

\begin{figure}
\epsfile{file=atg_dod2.ps,scale=0.8}
\caption{Average distances between start codons and ATG trinucleotides 
upstream/downstream}
\label{atg_dod2}
\end{figure}

\begin{figure}
\epsfile{file=atg_dist.ps,scale=0.7}
\caption{Average distances between start codons and specific
trinucleotides upstream}
\label{atg_dist}
\end{figure}

\begin{figure}
\epsfile{file=atg_disd.ps,scale=0.7}
\caption{Average distances between start codons and specific
trinucleotides downstream}
\label{atg_disd}
\end{figure}

The results show that for the upstream, the average distance
for AUG trinucleotide is the longest of all the other 
five trinucleotide patterns in eucaryotes(figure \ref{atg_dist}).
Furthermore, the average distances 
are generally longer in higher organisms
than in lower organisms; more specifically in the following order:
primates $>$ rodent $>$ mammals $>$ vertebrates $>$ invertebrates $>$
bacteria.  This rule  holds only  for AUG trinucleotides upstream,
not downstream(figure \ref{atg_dod2}).


\vspace{2ex}
\noindent
\begin{tabular}{|l|}
\hline
Discussion:\\
\hline
\end{tabular}

The result shows that there is a relationship between evolutions
and average distance upstream. 
One possible hypothesis we can
think for the reason 
that the distances of higher organisms are longer is as follows:
In bacteria, ribosomes bind directly to translation initiation site.
But in eucaryotes, ribosomes must first bind to CAP and look for
translation initiation site. Furthermore, bacteria mRNAs are
polycistronic and the coding regions may even overlap. And ribosomes
recognize translation initiation site accurately. From this point of
view, lower organisms 
may have more sophisticated and more reliable mechanism for translation 
initiation site. Thus ribosomes of lower organisms are unlikely to
confuse. 
We suggest that this is why distance between start codon and upstream 
AUG can be shorter in lower organisms.













\subsection{Discussion of the cases where two AUGs are close}

From our results, we suggest that there is evolutional pressure to 
eliminate AUG trinucleotides near start codons. This pressure is 
lower in lower organisms, presumably because lower organisms have
more reliable mechanisms to identify the appropriate AUG as a start codon.

However the existing AUGs, which are close to each other, does not
seem to have much influence on translation initiation. Thus the
hypothesis that upstream AUGs near start codons are
unpreferable holds in the long period of time. But in cases where two
AUGs are separated by two bases, strong consensus sequences may be 
needed to avoid leaky scannings.

\end{document}