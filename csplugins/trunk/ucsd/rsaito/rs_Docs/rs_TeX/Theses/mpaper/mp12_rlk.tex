
We have investigated the relationship between distances of two AUGs
and the rate of leaky scanning.

\vspace{2ex}
\noindent
\begin{tabular}{|l|}
\hline
Hypothesis:\\
\hline
\end{tabular}

When two AUGs are close to each other, ribosome confuse to select
one of AUGs as start codon. This phenomenon has influence on leaky
scanning. 
Thus rate of leaky scanning changes when two AUGs are close to each other.

\vspace{2ex}
\noindent
\begin{tabular}{|l|}
\hline
Data analysis:\\
\hline
\end{tabular}

The rate of leaky scanning when two AUGs are separated by \(n\) bases
(\(RLK_{n}\)) is calculated as follows.

\begin{small}
\begin{displaymath}
RLK_{n} = \frac{Number\:of\:leaky\:scanning\:observed\:when\:two\:AUGs\:
are\:separated\:by\:n\:bases}{Number\:of
\:cases\:that\:there\:is\:AUG\:trinucleotide\:n\:bases\:
upstream/downstream\:of\:start\:codon}
\end{displaymath}
\end{small}

\vspace{2ex}
\noindent
\begin{tabular}{|l|}
\hline
Results:\\
\hline
\end{tabular}

\begin{figure}
\begin{center}
\epsfile{file=leak_bun.ps,scale=0.40}
\end{center}
\caption{Relationship between leaky scanning and distance between two AUGs}
\label{leak_bun}
\end{figure}

The rate of leaky scanning fluctuates(figure \ref{leak_bun}), but this holds
not only when 
two AUGs are close to each other, but also when two AUGs are separated
far from each other.

While the first AUG is selected as a
starting AUG for the most of the time, we found that the second AUG is
frequently selected by leaky scanning if the two AUG's are
separated by \(3n+2\) bases where \(n\) is an integer.  

\vspace{2ex}
\noindent
\begin{tabular}{|l|}
\hline
Discussion:\\
\hline
\end{tabular}

The distance of two AUGs does have influence on rate of leaky
scanning, but we cannot observe any remarkable features when two AUGs
are close to each other. Thus, the hypothesis is not confirmed.

The periodic peak of the rate is presumably due to the following
hypothesis(figure \ref{3n2}).
Suppose that AUGs are separated by \(3n+2\) bases and translation
starts from first AUGs. The nucleotide pattern ``ua'' and ``ug''
just before second AUG trinucleotide and ``a'' just after the second 
AUG will make stop codons in frame, disturbing the farther translation
downstream. Thus, translation initiation from first AUG when second
AUG is separated by \(3n+2\) bases restricts these nucleotide patterns
around second AUGs. This restriction may make the rate of leaky scanning 
higher.

\begin{figure}
\begin{picture}(400,120)
\put(0,73){\line(1,0){50}}
\put(50,70){AUG}
\put(73,73){\line(1,0){200}}
\put(273,70){\underline{taA}UG}
\put(305,73){\line(1,0){50}}


\put(0,43){\line(1,0){50}}
\put(50,40){AUG}
\put(73,43){\line(1,0){200}}
\put(273,40){\underline{tgA}UG}
\put(305,43){\line(1,0){50}}



\put(0,13){\line(1,0){50}}
\put(50,10){AUG}
\put(73,13){\line(1,0){210}}
\put(282,10){A\underline{UGa}}
\put(311,13){\line(1,0){44}}

\put(150,80){$3n+2$ bases}
\put(240,81){\shortstack[c]{\small Stop codons in\\ the reading frames}}
\end{picture}

\caption{When two AUGs are separated by \(3n+2\) base pairs, stop
codons are likely to be in the same reading frame.}
\label{3n2}
\end{figure}






