
\section{Introduction}

Recently, a large amount of data on a variety of biological
experiments can be obtained freely through 
the internet. And as for species like {\it Mycoplasma
genitalium}\cite{label27},  {\it Haemophilas influenzae}, and yeasts, 
their whole DNA sequence data have already been read and we can easily 
obtain the data.
Obviously, amount of biological data such as DNA sequence data is increasing
drastically. On the other hand, amount of knowledge obtained from
these data is not increasing so fast.\\

\begin{center}
\epsfile{file=gengrow.ps,scale=0.6}\\
Figure: Increasing amount of biological data
\end{center}

To understand organisms, it is very important to 
find what meaning is hidden in these data. As the size of data is extremely
large, computer is very important tool for the fast analysis and knowledge
discovery. This new scientific field in which computer analyses of
biological data are done is called $Bio-informatics$. 

Biological experiments may investigate what kind of phenomenon 
occurs and how the phenomenon occurs in the specific cases. At the
same time, investigation
of organism by biological experiments has some disadvantages.

\begin{itemize}
\item Biological experiments are time consuming works, which often spend much
time to get single result.
When we want to know the universal law for a phenomenon, much data is needed.
But biological experiments take too much time.

\item There is the case where it is very difficult to see the cell and
its component at  
the molecular level. For example, protein is very important component for 
the cell, but it is very difficult to see its structure.
% Figure above shows number of entries for 
% GenBank(database for nucleic acid sequences)
% and PDB(database for structures of proteins) increasing each year. Although
% number of nucleic acid registered is increasing very rapidly, 
% number of structures
% of proteins registered is not increasing so fast.
Furthermore, looking the interaction of the component at the molecular level
is much more difficult.

\end{itemize}
 
$Bio-informatics$ investigates organisms from the view of information. 
By analyzing large amount of data in databases from various aspects, 
universal law for a phenomenon may be discovered. And by analyzing 
data which seems to be important for a specific interaction at the molecular
level, we can infer how the interaction occurs. Those
cannot be done by biological experiments. However, biological experiments are
important because it is only the way to directly observe the phenomenon.

So biological research must be done from two approaches to understand
 the organism.

\begin{itemize}
\item Approach from the experiments
\item Approach from the data analyses
\end{itemize}

In this paper, we approach to biology from the data analyses. The
topic is translation initiation.
The goal of this research is to find how the organisms recognize specific
DNA sequence as genes. Gene is information about how the organisms would
be like. This information is written on molecule called DNA in the cell.
But only part of DNA is used as genes. Then the question ``How do the 
organisms distinguish gene region of DNA?'' arises. The exact mechanism
is still unknown. If we discover new mechanisms, it must be the contribution
not only to biology, but also to biotechnology and medical.

We are conducting comprehensive computer analyses around boundaries
of gene regions. If the remarkable patterns are found, this may be 
influencing the distinction of gene region. 

In this paper, we focus on boundary called translation initiation site
and discuss the tendencies we discovered from this site.


\section{Purpose}

The goal of this research is to understand how the ribosomes 
determine the translation initiation site. In this paper, we use
comprehensive computer analyses, and try to discover tendencies that
 may be important for the selection of translation initiation site by 
ribosomes and infer the mechanism of them.

First, we conduct profile analyses and entropy calculations around
start codons to verify the hypotheses of
M.Kozak(eucaryotes)\cite{label3}, J.Shine and 
L.Dalgarno(procaryotes)\cite{label7}, and other researchers on translation
initiation. And further analyses are conducted to find remarkable tendencies.

Secondly, tendencies including frequencies of AUG trinucleotides
around start codons are invenstigated to support our original
hypothesis that ribosomes confuse if two AUGs are located close to
each other.

Thirdly, analyses of stop codons are conducted to investigate the
importance of reinitiation for leaky scanning, whose phenomenon is
investigated by M.Kozak\cite{label18} and other researchers. 

\section{Originality}

In the field of $bio-informatics$, there are many researches for making
the algorithm to analyze data. And there are many researches for analyzing
a specific biological substance or phenomenon such as analyses of protein 
function, analyses of third dimensional structure of protein, analyses
of metabolic pathway, analyses of evolution at the molecular model,etc. 
Databases are used to help their work.

However, there are few researches which are done by analyzing whole database
itself by creating original programs to find tendencies and infer the
biological mechanisms.
We have done it for translation initiation site from various points of view.


% \section{Significance of this Research}
% By analyzing translation initiation site from database, we may
% discover the mechanism of translation initiation that is not known
% so far. In the long term, this will make contribution not only to
% biology, but also to medical, and to biotechnology. 


\section{Results Expected}

By conducting comprehensive computer analyses of large amount of sequences
in database, we can expect the following results.

\begin{itemize}

\item We can get data that support the already known hypotheses on
translation initiation.
\item We can get data that is consistent with our original hypotheses
on translation initiation. 
\item We can discover tendencies by analyzing database from various
view points. From these discoveries, we can make new hypotheses.

\end{itemize}

% we can discover universal rule of ribosomes in determining
% the translation initiation site.

\section{Composition of This Paper}

Chapter \ref{matmeth} describes materials and methods for this research.
Chapter \ref{backgr} explains the foundation of genetics which is
needed to understand this paper(section \ref{founda}) and mechanism of
translation initiation discovered so far by other researches(section
\ref{surv}).  
Chapter \ref{resu} introduces our research and its results. 
In this chapter, section \ref{prof_ent} is mainly data analyses to
confirm the known 
hypotheses and section \ref{lowfreq} and section \ref{reinitia} are
mainly data analyses to confirm our original hypotheses.
Chapter \ref{concl} concludes our paper with future prospects.


