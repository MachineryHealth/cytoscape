
Frequencies of AUG trinucleotides around start codons are calculated
to find whether the frequencies decrease around start codons.

\vspace{2ex}
\noindent
\begin{tabular}{|l|}
\hline
Hypothesis:\\
\hline
\end{tabular}

Ribosomes confuse to select appropriate AUG as 
start codon if two AUGs are close to each other. Thus, the frequencies
of AUG trinucleotides around start codons are relatively low to
prevent the wrong selection of the AUGs.

\vspace{2ex}
\noindent
\begin{tabular}{|l|}
\hline
Data analysis:\\
\hline
\end{tabular}

Computer analysis was done to analyze the
frequencies of AUG trinucleotides around start codons. Except for
bacteria, mRNA data was used. For bacteria, DNA data was used, because
mRNA data for bacteria is very few. The frequency of position \(F_{p}\) is
calculated as follows.

\begin{displaymath}
F_{p} = \frac{Number\:of\:sequences\:that\:has\:AUG\:in\:position\:p}
      {Number\:of\:sequence\:data\:that\:has\:data\:for\:position\:p}
\end{displaymath}

And then smoothing was done for every 3 positions to reduce too much 
fluctuations. Because sequences with AUG trinucleotides located
upstream of start 
codons and in the same reading frame(every 3 positions from start
codon)are excluded for the precise analyses(described in
\ref{method_3}), we multiplied the smoothed  
value by \(\frac{3}{2}\) for upstream.

If the hypothesis holds, then the frequency of AUG trinucleotide gets
lower as the position in mRNA gets closer to start codon.
If not, the frequencies of AUG trinucleotides upstream will be constant for
upstream, and the frequencies of AUG trinucleotides downstream will
also be constant.

\vspace{2ex}
\noindent
\begin{tabular}{|l|}
\hline
Results:\\
\hline
\end{tabular}

\begin{figure}
\epsfile{file=pri_atg.ps,scale=0.40}
\epsfile{file=rod_atg.ps,scale=0.40}\\
\epsfile{file=mam_atg.ps,scale=0.40}
\epsfile{file=vrt_atg.ps,scale=0.40}\\
\epsfile{file=inv_atg.ps,scale=0.40}
\epsfile{file=bct_atg.ps,scale=0.40}
\caption{Frequencies of AUG trinucleotides around start codons}
\label{atg_freq}
\end{figure}

The results(figure \ref{atg_freq}) show that there is a general
tendency to have fewer  
AUG trinucleotides in front of start codons for all taxonomical
groups.
In eucaryotes, the frequencies of AUGs just downstream of start codon is low,
comparing to the frequencies far downstream, but the frequencies are
much higher comparing to ones in front of start codons.
In other words, the frequency drastically increases at the position
 of start codon for
all eucaryotes. But in bacteria, the frequencies of AUG trinucleotides
in positions just several nucleotides downstream of start codon is still low.

\vspace{2ex}
\noindent
\begin{tabular}{|l|}
\hline
Discussion:\\
\hline
\end{tabular}

The general tendency of the low frequency near start
codon in eucaryotes is consistent
with the hypothesis. The reason that the frequency drastically
increases at the position of start codon for all eucaryotes is
presumably
because according to the scanning model in eucaryotes,
ribosome scans mRNA from 5' end to downstream, searching for AUG.
If the AUG trinucleotide is located upstream of start codon, 
leaky scanning must work to prevent the initiation from this
inappropriate AUG. But if this mechanism is not reliable, 
those AUG trinucleotides should not exist upstream of start codons.  
But once translation starts
from start codon, AUG trinucleotide downstream of start codon has no 
influence on translation initiation.
Thus, this may be the reason that the frequencies of AUG
trinucleotides upstream of start codon is low and the frequencies
 of AUG trinucleotides downstream of start codon
is relatively high comparing to ones in front of start codons.

In bacteria, however, it is thought that ribosome binds directly to 
the translation initiation site instead of scanning mRNA from upstream
to downstream. Thus, we suggest that in bacteria, 
ribosome is likely to mistake nearby downstream AUGs for start codon
as well as nearby upstream AUGs.
