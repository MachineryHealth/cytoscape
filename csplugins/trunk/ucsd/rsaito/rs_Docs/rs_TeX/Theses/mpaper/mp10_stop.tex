Although cytosine or thymine in position -3 is known to promote leaky
scanning, our results showed that there is no remarkable consensus sequence to
promote leaky scanning. Then the question ``what promotes leaky scanning 
for the existing AUGs located upstream of start codons?'' arises.
It is known that leaky scanning occurs if
the first AUG is followed shortly by a stop codon(In other words
reinitiation occurs). In this section, we
focus on stop codons which is located downstream of the skipped
AUGs. 

\vspace{2ex}
\noindent
\begin{tabular}{|l|}
\hline
Hypothesis:\\
\hline
\end{tabular}

A lot of existing AUGs in 5'UTR are skipped by ribosomes by the stop
codons located just after in the same frame. Thus the frequency of
stop codons after the skipped AUGs in the same frame will be
relatively high. 


\vspace{2ex}
\noindent
\begin{tabular}{|l|}
\hline
Data analysis:\\
\hline
\end{tabular}

Frequencies of stop codons located downstream of first skipped AUGs
in position \(p\)(\(FS_{p}\)) are calculated by the following
formula(Position of A located in skipped AUG is defined +1.). 


\begin{displaymath}
FS_{p} =
\frac{Number\:of\:stop\:codons\:that\:appear\:in\:position\:p}{
Number\:of\:sequence\:data\:that\:has\:data\:for\:position\:p}
\end{displaymath}

We also calculated the rate of mRNAs that have stop codons located
downstream of the first skipped AUGs in the same frame and located
within specific distance \(d\)(\(RR_{d}\)) to investigate how many
skipped AUGs that can be explained by reinitiation mechanisms(Distance
\(d\) is defined as position \(d\)).

\begin{displaymath}
RR_{d} =
\frac{mRNAs\:whose\:first\:skipped\:AUGs\:have\:stop\:codons\:within\:distance\:d}{mRNAs\:that\:have\:skipped\:AUGs} 
\end{displaymath}

If a second AUG trinucleotide located
upstream of start codon is flanked by first AUG trinucleotide and stop 
codon in the same frame(in other words, if the second AUG
trinucleotide is located in the ``minicistron'' upstream of start
codon),  this AUG trinucleotide
may not have any effect on 
translation initiation. Thus, for the simplicity of the calculation, we
only focused on first skipped AUGs.
And we only calculated the cases where stop codons located downstream
of skipped AUG are located upstream of start codons, because the
influences to reinitiation by stop codons 
that are located downstream of start codons are unknown.


\vspace{2ex}
\noindent
\begin{tabular}{|l|}
\hline
Results:\\
\hline
\end{tabular}

\begin{figure}
\begin{center}
\epsfile{file=eu_stopf.ps,scale=0.40}
\end{center}
\caption{Frequencies of stop codons located downstream of the first
skipped AUGs and in the same frame as the first skipped AUGs}
\label{eu_stopf}
\end{figure}


\begin{figure}
\epsfile{file=gbstop.ps,scale=0.67}
\caption{Frequencies of stop codons located downstream of the first
skipped AUGs in vertebrates}
\label{gbstop}
\end{figure}

\begin{figure}
\epsfile{file=invstop.ps,scale=0.67}
\caption{Frequencies of stop codons located downstream of the first
skipped AUGs in invertebrates}
\label{invstop}
\end{figure}

\begin{figure}
\begin{center}
\epsfile{file=stop_reini.ps,scale=0.40}
\end{center}
\caption{Rate of mRNAs that have stop codons located downstream of
the first skipped AUGs in the same frame within specific distance}
\label{stop_reini}
\end{figure}

Figure \ref{eu_stopf} shows frequency of stop 
codons after the first skipped
AUGs in the same reading frame.  The
stop codon frequencies gradually decrease in
both vertebrates and invertebrates.  Also according to the figure
\ref{gbstop} and \ref{invstop}, there are regular peaks of
the frequency every 3 base positions(corresponding bar is colored in
 black in the figure)
 from the skipped AUG.  In
other words, stop codons exist preferably in the same reading frame
as the skipped AUG. 

Figure \ref{stop_reini} shows rate of mRNAs that have stop codons
located downstream of the first skipped AUGs in the same frame and within
specific distance from the first skipped AUGs. 



\vspace{2ex}
\noindent
\begin{tabular}{|l|}
\hline
Discussion:\\
\hline
\end{tabular}

From these results, stop codons located downstream 
and close to AUG trinucleotides may be important for the leaky scannings.
From figure \ref{stop_reini}, if
the ribosomes have ability to reinitiate translation when stop codons
are located within position +20 from AUG trinucleotides, 
\(\frac{1}{4}\) of the cause of leaky scanning may be explained by
reinitiation mechanism. Although it is believed that this distance
must be short, we cannot suggest the appropriate
distance. In human immunodeficiency virus type 1 mRNA, if the upstream 
open reading frame consists of 84 nucleotides, reinitiation occurs by
50\%\cite{label31}.  
% In Kozak's paper, it is written that reinitiation occurs
% when AUG trinucleotide is followed shortly by stop codons.



