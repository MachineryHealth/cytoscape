In this subsection, investigation about whether there is any
relationship between evolution and translation initiation mechanism
will be conducted.

\vspace{2ex}
\noindent
\begin{tabular}{|l|}
\hline
Hypothesis:\\
\hline
\end{tabular}

There is a relationship between the evolution and distance between
start codon and upstream AUG trinucleotide.

\vspace{2ex}
\noindent
\begin{tabular}{|l|}
\hline
Data analysis:\\
\hline
\end{tabular}

We have calculated the average distances between a start codon and the nearest
non-starting ATG trinucleotide to the upstream and to the
downstream. This time, we have used DNA data rather than mRNA data in
GenBank because a large amount of data are required for precise
analyses and there are more sequences as DNA data than as mRNA data.
For the sake of comparison, distances between  start codons and other
trinucleotides consisting of 'A','T','G' are measured.

\vspace{2ex}
\noindent
\begin{tabular}{|l|}
\hline
Results:\\
\hline
\end{tabular}

\begin{figure}
\epsfile{file=atg_dod2.ps,scale=0.8}
\caption{Average distances between start codons and ATG trinucleotides 
upstream/downstream}
\label{atg_dod2}
\end{figure}

\begin{figure}
\epsfile{file=atg_dist.ps,scale=0.7}
\caption{Average distances between start codons and specific
trinucleotides upstream}
\label{atg_dist}
\end{figure}

\begin{figure}
\epsfile{file=atg_disd.ps,scale=0.7}
\caption{Average distances between start codons and specific
trinucleotides downstream}
\label{atg_disd}
\end{figure}

The results show that for the upstream, the average distance
for AUG trinucleotide is the longest of all the other 
five trinucleotide patterns in eucaryotes(figure \ref{atg_dist}).
Furthermore, the average distances 
are generally longer in higher organisms
than in lower organisms; more specifically in the following order:
primates $>$ rodent $>$ mammals $>$ vertebrates $>$ invertebrates $>$
bacteria.  This rule  holds only  for AUG trinucleotides upstream,
not downstream(figure \ref{atg_dod2}).


\vspace{2ex}
\noindent
\begin{tabular}{|l|}
\hline
Discussion:\\
\hline
\end{tabular}

The result shows that there is a relationship between evolutions
and average distance upstream. 
One possible hypothesis we can
think for the reason 
that the distances of higher organisms are longer is as follows:
In bacteria, ribosomes bind directly to translation initiation site.
But in eucaryotes, ribosomes must first bind to CAP and look for
translation initiation site. Furthermore, bacteria mRNAs are
polycistronic and the coding regions may even overlap. And ribosomes
recognize translation initiation site accurately. From this point of
view, lower organisms 
may have more sophisticated and more reliable mechanism for translation 
initiation site. Thus ribosomes of lower organisms are unlikely to
confuse. 
We suggest that this is why distance between start codon and upstream 
AUG can be shorter in lower organisms.











