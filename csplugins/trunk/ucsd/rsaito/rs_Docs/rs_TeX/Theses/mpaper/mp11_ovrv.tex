
\section{Flow}

We make some hypothesis on translation initiation and its sites. Then
we access database to get data and calculate, using computer programs
we develop, to check whether
the results are consistent with our hypothesis or not.
Researches are done by the flow shown in figure \ref{res_flow}.

\begin{itemize}
\item Decide topic\\
Decide what topic to investigate. In this paper, the whole topic is
translation initiation.

\item Survey about topic\\
Survey the researches that are done on the decided topic.

\item Make hypotheses\\
From the investigation done by other researchers so far,
make hypotheses on the topic.

\item Conduct data analyses\\
Make analyses using database to investigate whether the hypotheses hold
or not. To analyze the database, write computer programs.

\item Analyze results\\
Analyze the output from the computer programs.

\item Discussion\\
Examine whether the output matches to the hypotheses or not.
Depending on the results, further analyses will be done.

\end{itemize}

\begin{figure}
\begin{picture}(400,220)

\put(100,210){\framebox(150,10){Decide topic}}
\put(175,205){\vector(0,-1){20}}

\put(100,170){\framebox(150,10){Survey about topic}}
\put(175,165){\vector(0,-1){20}}

\put(100,130){\framebox(150,10){Make hypotheses}}
\put(175,125){\vector(0,-1){20}}

\put(100,90){\framebox(150,10){Conduct data analyses}}
\put(175,85){\vector(0,-1){20}}

\put(100,50){\framebox(150,10){Analyze results}}
\put(175,45){\vector(0,-1){20}}

\put(255,15){\line(1,0){100}}
\put(355,15){\line(0,1){160}}
\put(355,175){\vector(-1,0){100}}
\put(355,135){\vector(-1,0){100}}

\put(100,10){\framebox(150,10){Discussion}}

\end{picture}
\caption{Flow of the research}
\label{res_flow}
\end{figure}

\section{Method for data analyses}\label{method_3}
For the data analyses, GenBank database release 94.0 can be used.
In this database, sequences are divided into some taxonomical groups.
Followings are the files we used.

\begin{description}
\item[gbpri.seq] Primate sequence entries
\item[gbrod.seq] Rodent sequence entries.
\item[gbmam.seq] Other mammalian sequence entries.
\item[gbvrt.seq] Other vertebrate sequence entries.
\item[gbinv.seq] Invertebrate sequence entries.
\item[gbbct.seq] Bacterial sequence entries.
\end{description}

Those files are installed as UNIX files and accessible from C
language. Thus for the analyses, we created original C programs to get
specific data and analyze them. 

\begin{figure}
\begin{picture}(400,30)
\put(90,0){\framebox{\shortstack[c]{GenBank\\ database}}} 
\put(190,5){\framebox{Original C Programs}}
\put(350,5){\framebox{Results}}

\put(150,9){\vector(1,0){30}}
\put(305,9){\vector(1,0){30}}
\end{picture}
\caption{Method for data analysis}
\end{figure}

We used mRNA data for the analysis, because mRNA is involved in
translation much more than DNA is. But as mRNA data for bacteria
are few, we used DNA data instead for bacteria.
Also when we calculate average distances between start codon and
nearest upstream AUG trinucleotides, we used DNA for the precise
analyses because there are more DNA data than mRNA data.

And following data are excluded from our analyses.
\begin{itemize}
\item Pseudo genes. Pseudo genes are not translated in vivo. Computer
program eliminates data in the entry which  has keyword
``pseudo'' either in DEFINITION or in CDS field.
\item Immunoglobulin and receptor sequences. They have special translation
features. Computer program eliminates data in the entry which has keyword
``immuno'',''receptor'' or ``variabl'' in DEFINITION field. 
\item Mitochondrial sequences except for the mitochondria analyses.
Mitochondria have special translation initiation features. Computer
program eliminates data in the entry which has keyword ``mitochond''
in DEFINITION field. 
\item Partial sequences. Computer program eliminates data in the entry 
which has keyword ``partial'' in either DEFINITION field or CDS field.
Also it eliminates entries with keyword ``exon'' in DEFINITION field. 
\item Sequences that have introns in 5'UTR. Computer program
recognizes the location of intron by the keyword ``intron'' and the
location following to that keyword.
\item Putative sequences. Computer program eliminates keyword
``/note="putative"'' in CDS field.
\item Sequences that have non-AUG start codon in eucaryotes. Since
eucaryotes almost always initiate translation from AUG codon in vivo, we
only accept data with AUG start codon. 
\item AUG trinucleotide in the same frame as start codon in 5'UTR.
Start codons in the database are likely to be wrong when two AUGs are
in the same reading frame. Thus, for the precise analysis, we
eliminated those data.


\end{itemize}




